\section{Project structure}
You can find the source code \href{https://github.com/pinzauti/automated-reasoning}{here} and the source for the report \href{https://github.com/pinzauti/automated-reasoning-report}{here}.

The file structure of the project is the following\graffito{In this section we try to give a quick look at the whole project, please note that if further details are needed all the code is properly documented, both the models and the Python program.}:
\dirtree{%
.1 src.
.2 asp\DTcomment{ASP part of the project.}.
.3 data\DTcomment{Input data.}.
.4 input.lp\DTcomment{User defined input for manual mode.}.
.4 benchmark1.lp\DTcomment{Benchmark instance.}.
.4 \dots.
.4 benchmark30.lp\DTcomment{Benchmark instance.}.
.3 model.lp\DTcomment{ASP Model.}.
.2 minizinc\DTcomment{MiniZinc part of the project.}.
.3 data\DTcomment{Input data.}.
.4 input.dzn\DTcomment{User defined input for manual mode.}.
.4 benchmark1.dzn\DTcomment{Benchmark instance.}.
.4 \dots.
.4 benchmark30.dzn\DTcomment{Benchmark instance.}.
.3 model.mzn\DTcomment{MiniZinc Model.}.
.2 main.py.\DTcomment{Program entrypoint.}.
.2 asp.py.\DTcomment{Handles everything related to ASP.}.
.2 mz.py.\DTcomment{Handles everything related to MiniZinc.}.
.2 Makefile\DTcomment{Define commands to setup and execute the project.}.
.2 requirements.txt\DTcomment{Required packages to run the project.}.
}

\subsection{Tech stack}

Everything was developed, tested and executed on a local machine\footnote{OEM: Xiaomi, OS: Windows 11 21H2, CPU: Intel Core i7-11370H, RAM: 16GB DDR4, GPU: NVIDIA GeForce MX450.}. The project is built using \href{https://www.python.org/downloads/release/python-3102/}{Python 3.10.2}, which allows to manage the ASP and MiniZinc models, the random instances and their solutions in only one command line interface.

\href{https://github.com/potassco/clingo/releases/tag/v5.5.2}{Clingo 5.5.2} and a \href{https://marketplace.visualstudio.com/items?itemName=abelcour.asp-syntax-highlight}{Visual Studio Code extension} was used for ASP. All the project is built upon Clingo API~\cite{ClingoAPI}, which was necessary in order to generate and solve random instances and to format the output in an human readable way.

As far as MiniZinc goes the model was developed using \href{https://github.com/MiniZinc/libminizinc/releases/tag/2.6.3}{MiniZinc 2.6.3} and \href{https://www.minizinc.org/ide/}{MiniZinc IDE}. Again, MiniZinc API~\cite{MiniZincPython} was heavily used in the main script.

In both cases \href{https://copilot.github.com/}{Github Copilot} assisted the development process.

This report is written using \href{https://www.latex-project.org/}{LaTeX}, the \href{https://ctan.org/pkg/minted?lang=en}{Minted package} for the source code and \href{https://inkscape.org/}{Inkscape} for creating the figures.


\subsection{How to run}
First thing first you need to create a virtual environment and install the required packages\footnote{It will detect automatically if you are using Windows or Linux, no need to adjust anything.}. To do so just run the following commands\graffito{The packages installed are, peraphs unsuprisingly, clingo and minizinc[dzn].}:
\begin{minted}{bash}
    cd src/
    make setup
\end{minted}

If you want to know what you can do just type:

\begin{minted}{bash}
    make help 
\end{minted}

We will however explain everything here.

\subsubsection{ASP}
Two modes are available: a random mode and a manual mode. The random mode will generate a random instance and solve it. The manual mode will allow you to define your own instance and solve it.

\paragraph{Random mode}

If you want to use random mode you need to run the following command:
\begin{minted}{bash}
    make asp-random  [d DIMENSION] [n NUMBER] [s SOLUTIONS]
\end{minted}

you can choose the dimension of the board and the number of initial points. However if you don't want to the program will generate them randomly for you; you can also decide to input only the dimension or only the number of starting points. Note that a upper bound was set on the dimension, both if you manually define it or if you don't. Going beyond that could lead to excessive execution time\footnote{Why did you not used the --time-limit option? \href{https://github.com/potassco/clingo/issues/151}{The API does not seem to support it}. However all the benchmarks in section~\ref{sec:benchmarks} respect the 5 minutes time limit as required.}.
By default all solutions will be displayed, if you want to display only a specific amount of solution specify it to the argument \mintinline{bash}{s}. 

All of this is managed by the \mintinline{python}{asp_random()} function in the \file{main.py} file. 
It goes without saying that not all the randomly generated instances will be solvable, and the program will communicate when those are not.

\paragraph{Manual mode}

If you want to use manual mode you need to run the following command:
\begin{minted}{bash}
    make asp-manual [s SOLUTIONS]
\end{minted}
this will solve the problem with the input data that you should enter in the file \path{src\asp\data\input.lp}. The \mintinline{bash}{s} argument is the same as explained above. All of this is managed by the \mintinline{python}{asp_manual()} function in the \file{main.py} file.
Some sanity checks on the user input are performed, but it should still be reasonable in order to produce something meaningful and in an acceptable time.

As anticipated the solutions will be printed in an human readable way thanks to the function \mintinline{python}{asp_prettier()} located in the \file{main.py} file.

\subsubsection{MiniZinc}
Again, two modes are available: a random mode and a manual mode. The parameters and the functioning are the same, however we briefly present them\graffito{If the program returns you some kind of generic error it is most likely that you should add MiniZinc installation location to your PATH. More on that \href{https://github.com/MiniZinc/minizinc-python/issues/60}{here}.}.

\paragraph{Random mode}

If you want to use random mode you need to run the following command:
\begin{minted}{bash}
    make minzinc-random  [d DIMENSION] [n NUMBER] [s SOLUTIONS]
\end{minted}
where \mintinline{bash}{d} is the dimension of the board, \mintinline{bash}{n} is the number of starting points and \mintinline{bash}{s} is the number of solutions to be displayed. All the parameters are optional, if not inserted  \mintinline{bash}{d} and  \mintinline{bash}{n} will be randomly chosen by the program, and all solution will be displayed.

\paragraph{Manual mode}

If you want to use manual mode you should insert the input data in the file \path{src\minizinc\data\input.dzn} and run the following command:
\begin{minted}{bash}
    make minizinc-manual [s SOLUTIONS]
\end{minted}
where \mintinline{bash}{s} is the number of solutions to be displayed.

\subsubsection{Both}
You can also solve both the ASP and the MiniZinc problems at the same time, inserting the input data in the file \path{src\asp\data\input.lp} and \path{src\minizinc\data\input.dzn}. The command is
\begin{minted}{bash}
    make both [s SOLUTIONS]
\end{minted}
where \mintinline{bash}{s} as always is the number of solutions to be displayed.