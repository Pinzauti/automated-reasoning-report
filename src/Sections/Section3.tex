\section{ASP model}
As explained in~\cite{ClingoGuide} an ASP model is divided in four part: generate, define, test and display.
As for the generate part our Python program handles both the generation of the random instances in random mode and the inclusion of the user defined input in manual mode.
\subsection{Generate}
The input is composed by \mintinline{text}{n} + 1 ground terms, with \mintinline{text}{n} the number of starting points.
\begin{minted}[breaklines, linenos]{text}
    boardl(D).
    number(X_1, Y_1, T_1).
    .
    .
    .
    number(X_n, Y_n, T_n).
\end{minted}
\mintinline{text}{D} is the dimension of the board and \mintinline{text}{X_i, Y_i} the coordinates of the starting points. \mintinline{text}{T} is the number of turns each starting point has to make to arrive to the intersection point.

\subsection{Define}
The define part is displayed below
\begin{minted}[breaklines, linenos]{text}
    grid(1..D, 1..D) :-boardl(D).
    1 {goal(X,Y): grid(X,Y)} 1.
    
    {link(start(X1, Y1), end(X2, Y2)) :  grid(X1,Y1), grid(X2,Y2), |X1 - X2| + |Y1 - Y2| = 1}. 
    
    connected(start(X1,Y1) ,end(X2,Y2)) :- link(start(X1,Y1), end(X2,Y2)).
    connected(start(X1,Y1), end(X2,Y2)) :- connected(start(X1,Y1), end(Z1,Z2)), link(start(Z1,Z2),end(X2,Y2)). 
\end{minted}

In line 1 we have a rule for defining the predicate grid, i.e. a matrix of dimension $DxD$. The goal point, i.e. the intersection point of our links is defined in line 3: this has to be one and has of course to be part of the grid.
On line 4 we have a rule for defining the predicate link, which is the connection between two points of the grid, this connection can take place or not and can't be diagonal but only horizontal and vertical. Note the use of the predicates \mintinline{text}{start} and \mintinline{text}{end} which serves the scope to give directionality to the links.
On line 5 we have the connected predicate, defined recursively on the link definition, which is the predicate that checks if two points are connected.
\subsection{Test}
Let us start describing the first denials:

\begin{minted}[breaklines, linenos]{text}
    :- #count{grid(X,Y): number(X,Y, _)} < 2. 

    :- goal(X, Y), number(X, Y, _). 
    
    :- goal(X, Y), not connected(start(X0, Y0), end(X,Y)), number(X0,Y0, _). 
\end{minted}

In line 1 we impose the number of starting point to be at least 2. In line 2 we impose that the goal point must not be a starting point.
In line 3 we impose that the goal point must be connected to a starting point.


\begin{minted}[breaklines, linenos]{text}
    :- link(start(X1, Y1), end(X0, Y0)), number(X0, Y0, _). 

    :- link(start(X,Y), _), not link(_, end(X,Y)), not number(X,Y, _).
    
    :- link(_, end(X,Y)), not link(start(X,Y), _), not goal(X, Y). 
\end{minted}
In line 1 we impose that no link can end in a starting point. In line 2 we impose that a link can start in a point only if in that point another link ends or if it is the starting point. In line 3 we impose that a link can end in a point only when in that point another link starts or if it is the goal.


\begin{minted}[breaklines, linenos]{text}
    :- link(start(X, Y),end(X1, Y1)), link(start(X, Y), end(X2, Y2)), X1 != X2. 

    :- link(start(X, Y),end(X1, Y1)), link(start(X, Y), end(X2, Y2)), Y1 != Y2. 
    
    :- link(start(X1, Y1),end(X, Y)), link(start(X2, Y2), end(X, Y)), X1 != X2, not goal(X,Y). 
    
    :- link(start(X1, Y1),end(X, Y)), link(start(X2, Y2), end(X, Y)), Y1 != Y2, not goal(X,Y). 
\end{minted}

In line 1 and 2 we impose that there can not be two links from one starting point. In line 3 and 4 we impose that two links can't end in the same point (i.e. there can not be intersections).
\begin{minted}[breaklines, linenos]{text}
    :- link(start(X1, Y1), end(X2, Y2)), link(start(X2,Y2), end(X1, Y1)).

    :- link(_, end(X1,Y1)), not connected(start(X1,Y1), end(X, Y)), goal(X, Y), not goal(X1,Y1). 
\end{minted}

In line 1 we  impose that a link can not end when it starts. In line 2 we impose that a link have to be connected to the goal. All together, we are avoiding loops.

\begin{minted}[breaklines, linenos]{text}
    :- #count{grid(X2,Y2), grid(X3, Y3): link(start(X1, Y1), end(X2, Y2)), link(start(X2, Y2), end(X3, Y3)), |X1-X3| = 1, |Y1-Y3| = 1, connected(start(X0, Y0), end(X1, Y1));
    grid(X1,Y1), grid(X2, Y2): link(start(X0, Y0), end(X1, Y1)), link(start(X1, Y1), end(X2, Y2)), |X0-X2| = 1, |Y0-Y2| = 1} != T, number(X0, Y0, T).
\end{minted}

Finally on the last line we impose that from each starting point to the goal the number of turns should be the one prescribed. We do that by checking a triple of consecutive links, if the difference between the x-axis and y-axis coordinates is 1 for both, then we have a turn.
 Note that we make use of \emph{pooling} in order to consider two cases. The first case is when the turn is made by a a triple of links which does not start from a starting point (but they must be obviously connected to it), the second case is when the turn is made by a triple directly starting from a starting point.

\subsection{Display}

Finally with the last two lines we display our solution, hiding everything except the goal and the link towards it.