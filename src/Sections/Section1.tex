\section{The problem}

\section{Project structure}
You can find the source code \href{https://github.com/pinzauti/automated-reasoning}{here} and the source for the report \href{https://github.com/pinzauti/automated-reasoning-report}{here}

\subsection{Stack used}

Everything was developed, tested and executed on a local machine\footnote{OEM: Xiaomi, OS: Windows 11 21H2, CPU: Intel Core i7-11370H, RAM: 16GB DDR4, GPU: NVIDIA GeForce MX450.}. As an entrypoint to both ASP and MiniZinc models, as well as instances initialization \href{https://www.python.org/downloads/release/python-3102/}{Python 3.10.2} was used.

\href{https://github.com/potassco/clingo/releases/tag/v5.5.2}{Clingo 5.5.2} and a \href{https://marketplace.visualstudio.com/items?itemName=abelcour.asp-syntax-highlight}{Visual Studio Code extension} was used for ASP, while \href{https://github.com/MiniZinc/libminizinc/releases/tag/2.6.3}{MiniZinc 2.6.3} and \href{https://www.minizinc.org/ide/}{MiniZinc IDE} for MiniZinc. \href{https://copilot.github.com/}{Github Copilot} was also used.

The file structure of the project is the following:
\dirtree{%
.1 src.
.2 asp\DTcomment{ASP part of the project.}.
.3 data\DTcomment{Input data.}.
.4 input.lp\DTcomment{User defined input for manual mode.}.
.3 main.lp\DTcomment{ASP Model.}.
.2 minizinc\DTcomment{MiniZinc part of the project.}.
.3 data\DTcomment{Input data.}.
.4 input.mzn\DTcomment{User defined input for manual mode.}.
.3 main.mzn\DTcomment{MiniZinc Model.}.
.2 main.py.
.2 Makefile\DTcomment{Define commands to setup and execute the project.}.
.2 requirements.txt\DTcomment{Required packages to run the project.}.
}

\begin{minted}{bash}
    make setup
\end{minted}

If you want to know the commands available just type

\begin{minted}{bash}
    make help 
\end{minted}
\section{ASP}

\section{Minizinc}

\section{Benchamrks}