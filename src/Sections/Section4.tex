\section{MiniZinc model}
The approach taken with MiniZinc is completely different. Data structures used to represent the same data and relations are different, input and output format are completely different.
\subsection{Data structures}
Let us introduce everything with order:
\begin{minted}[breaklines, linenos]{minizinc}
par int: dimension; set of int: D = 1..dimension; 
par int: number; set of int: N = 1..number; 
array[N, 1..3] of par 0..dimension: starting_points; 
array[D, D, 1..2] of var 0..12: Board;
\end{minted}

On line 1 we define the dimension of the board as a parameter named, and a set of integers \mintinline{minizinc}{N} on top of that dimension.
After that, on line 2, we need to define also the number of starting points\footnote{This is something not necessaries in ASP, but it is necessary in MiniZinc as we are using a multidimensional array to represent the starting points and MiniZinc does not support dynamically sized arrays.}. and again a set of int \mintinline{minizinc}{N} on top of that.
On line 3 we define the input format of the starting points which will be of the form:
\begin{minted}[breaklines, linenos, escapeinside=@@]{minizinc}
starting_points = [|X@$_1$@, Y@$_1$@, T@$_1$@
                    ...
                    |X@$_n$@, Y@$_n$@, T@$_n$@|];
\end{minted}
with  \mintinline{minizinc}{n} the number of starting points and each line representing, in order, the $X$, $Y$ coordinates of a starting point and the number of turns it has to make in order to arrive to the goal.

Finally, on line 3, we define the multidimensional array of variables \mintinline{minizinc}{Board}. \mintinline{minizinc}{Board} is of size $D$x$D$ and has two dimensions: the first one will store in each position one of the possible elements of the board among the one represented in figure~\ref{figm:board}:
\begin{figure}[h!]
    \centering
    \def\svgwidth{\columnwidth}
    \import{./Images/}{minizinc.pdf_tex}
    \caption{First dimension of Board. Each number corresponds to the element represented.}
    \label{figm:board}
\end{figure}
the second dimension of the board will keep track of which path\footnote{We have one path for each starting point, all paths intersect in one single point at the end.} each cell belongs to, this is necessary, as we will se later, in order to constraint the number of turns each path has to make.

Below you see the matrix Board for the given example\footnote{The second dimension of the Board is actually hidden in the output as not really relevant and only useful for internal checks. However if you want to show it it is sufficient to replace \mintinline[breaklines]{minizinc}{show(Board[i,j,1]) | i,j in D} with \mintinline[breaklines]{minizinc}{show(Board[i,j,1]) | t in 1..2, i,j in D}.}:


\mintinline{minizinc}{Board[i,j,1]}:
\begin{center}
    \begin{tabular}{ |c|c|c| } 
     \hline
     0 &0 &12 \\
     \hline
4 &5 &9 \\
\hline
12 &12 &1 \\
     \hline
    \end{tabular}
    \end{center}

\mintinline{minizinc}{Board[i,j,2]}:
    \begin{center}
    \begin{tabular}{ |c|c|c| } 
        \hline
        0 &0 &3 \\
        \hline
   1 &1 &4 \\
   \hline
   1 &2 &2 \\
        \hline
       \end{tabular}
    \end{center}
Note that the cell labelled, in the second dimension of the board, as \mintinline{minizinc}{number + 1} (in this example 4) is the cell containing the intersection of all the paths.

\subsection{Predicates}\label{subsec:predicates}
We start defining the predicates \mintinline{minizinc}{t_cells(i,j)},\mintinline{minizinc}{b_cells(i,j)},\mintinline{minizinc}{l_cells(i,j)} and \mintinline{minizinc}{r_cells(i,j)}. These predicates defines the possible cells, again among the ones in figure~\ref{figm:board}, that we can find on the top, on the bottom, on the left and on the right.
Let us take an example of one of this predicate:
\begin{minted}[breaklines, linenos]{minizinc}
predicate t_cells(int: i, int: j) =
(Board[i-1, j, 1] = 2 \/ Board[i-1, j, 1] = 4 \/ Board[i-1, j, 1] = 6 \/ Board[i-1, j, 1] = 8 \/ Board[i-1, j, 1] = 9 \/ Board[i-1, j, 1] = 10 \/ Board[i-1, j, 1] = 11);
\end{minted}

On top of that we build the predicates \mintinline{minizinc}{top(i,j)},\mintinline{minizinc}{bottom(i,j)},\mintinline{minizinc}{left(i,j)} and \mintinline{minizinc}{right(i,j)}.

Let us take an example of one of those predicates:
\begin{minted}[breaklines, linenos]{minizinc}
predicate top(int: i, int: j) =
i-1 > 0 /\ 
(t_cells(i,j)\/ Board[i-1, j, 1] = 12) /\ 
(Board[i-1, j, 2] != number + 1 -> (Board[i, j, 2] = Board[i-1, j, 2]  \/ Board[i, j, 2] = number + 1));
\end{minted}
We are in position \mintinline{minizinc}{i}, \mintinline{minizinc}{j} and we want to state the possible elements on top of that cell. First of all we check to be still inside the board, i.e. that the position $i-1$ is greater than zero. Then we proceed allowing all the cells that can start on top of another cell (in this example 2,4,6,8,9) and this is done in \mintinline[breaklines]{minizinc}{t_cells(i,j)}. This was for the first dimension of the board. As far as the second dimension we check that the top cell is not an intersection cell, the intersection cell is labelled on the second board with the number \mintinline{minizinc}{dimension + 1}. If it is not we required the cell \mintinline{minizinc}{i}, \mintinline{minizinc}{j} to be on the same path of the cell \mintinline{minizinc}{i-1}, \mintinline{minizinc}{j} or the be an intersection point. In this way we keep track of each path, and allow them to end in the same point.

\subsection{Constraints}
We shall start placing in each starting point a cell of type 12 (again, according to figure~\ref{figm:board} which from now on will be taken as reference when stating any cell number). Note that in order to be as consistent as possible with the ASP model we have to adapt the peculiar notation of the assignment (i.e. inverse row index, column and row notation inverted) to the MiniZinc notation. The code does so:

\begin{minted}[breaklines, linenos]{minizinc}
forall(n in N)(
if starting_points[n,2] = 1 then Board[dimension, starting_points[n,1], 1] = 12 /\ Board[dimension, starting_points[n,1], 2] = n
elseif
starting_points[n,2] = dimension then Board[1, starting_points[n,1], 1] = 12 /\ Board[1, starting_points[n,1], 2] = n
else
Board[dimension - starting_points[n,2] + 1, starting_points[n,1], 1] = 12 /\ Board[dimension - starting_points[n,2] + 1, starting_points[n,1], 2] = n
endif
)
\end{minted}
In the first dimension of the board we place the number 12 while in the second dimension we label each starting point with a number, this number will define each path on the board.

We proceed now using the predicates of~\ref{subsec:predicates} in order to define which moves are allowed for each type of cell.


\begin{minted}[breaklines, linenos]{minizinc}
forall(i,j in D)(

    (Board[i, j, 1] = 1 -> left(i,j) /\ top(i,j)) /\

    ...

    (Board[i, j, 1] = 12 -> top(i,j) \/ bottom(i,j) \/ right(i,j) \/ left(i,j))

)
\end{minted}

We reported only two example, the cell 1 allow to move on the left and on top, while the starting point cell allows to move everywhere. All the other cells allowed moves are properly defined.

Another constraint that we impose (and the code of which we omit, as it is quite trivial) is imposing that if a cell is empty in the first dimension of the board it should be the same even in the second dimension and viceversa.

In addition we impose that on the second dimension of the board there can't be number (i.e. paths) greater than \mintinline[breaklines]{minizinc}{number + 1}.

The last constraint avoid multiple paths from a single starting point, we partially report it below:

\begin{minted}[breaklines, linenos]{minizinc}
forall(i,j in D where Board[i,j,1] = 12) (
if top(i,j) then if i+1 <= dimension then not b_cells(i,j) endif /\ if j-1 > 0 then not l_cells(i,j) endif /\ if j+1 <= dimension then not r_cells(i,j) endif

...

elseif right(i,j) then if i-1 > 0 then not t_cells(i,j) endif /\ if i+1 <= dimension then not b_cells(i,j) endif /\ if j-1 > 0 then not l_cells(i,j) endif
else true endif
)
\end{minted}


\subsection{Global constraints}
The following global constraints were used:

\begin{description}
    \item[among] \mintinline[breaklines]{minizinc}{among(var int: n, array [$X] of var int: x, set of int: v)} Requires exactly \mintinline{minizinc}{n} variables in \mintinline{minizinc}{x} to take one of the values in \mintinline{minizinc}{v}.
    \item[count\_eq]  \mintinline[breaklines]{minizinc}{predicate count_eq(array [$X] of var int: x, var int: y, var int: c)} Constrains \mintinline{minizinc}{c} to be the number of occurrences of \mintinline{minizinc}{y} in \mintinline{minizinc}{x}.
\end{description}
The use of let expression was also require, we report the syntax below:

\begin{minted}[breaklines]{minizinc}
<let-expr> ::= "let" "{" <let-item> ";" ... "}" "in" <expr>

<let-item> ::= <var-decl-item>
             | <constraint-item>
\end{minted}

Let us explain how global expressions were used starting with:

\begin{minted}[breaklines, linenos]{minizinc}
let {array[D,D] of var 0..12: MoveBoard; constraint forall(i,j in D) (MoveBoard[i,j] = Board[i,j, 1]);} in
if number = 2 then
among(0, MoveBoard, 7..11)
elseif number = 3 then
among(1, MoveBoard, 7..10) /\ count_eq(MoveBoard, 11, 0)
elseif number = 4 then
count_eq(MoveBoard, 11, 1) /\ among(0, MoveBoard, 7..10) 
else
false 
endif
\end{minted}
Here we require that, depending on the number of starting points, there are a certain type of intersection cells, e.g. if we have four points we need a single interception cell, the number 11.
And so on for the other cases.

We proceed now with three brief constraints\graffito{Some of those constraints could seem redundant, but for how the predicates in section~\ref{subsec:predicates} were defined they turned to be necessary.}:
\begin{minted}[breaklines, linenos]{minizinc}
constraint(
forall (n in N)(
let {array[D,D] of var 0..12: MoveBoard; constraint forall(i,j in D) (if  Board[i,j,2] = n then MoveBoard[i,j] = Board[i,j, 1] else MoveBoard[i,j] = 0 endif);} in
among(starting_points[n,3], MoveBoard, 1..4)
));

constraint(
let {array[D,D] of var 0..12: PathsBoard; constraint forall(i,j in D) (PathsBoard[i,j] = Board[i,j,2]);} in
count_eq(PathsBoard, number + 1, 1)
);

constraint(
let {array[D,D] of var 0..12: MoveBoard; constraint forall(i,j in D) (MoveBoard[i,j] = Board[i,j,1]);} in
count_eq(MoveBoard, 12, number)
);
    \end{minted}

In the first one we require each path to do the prescribed number of turns, in the second one we impose that the number of intersection point (i.e. point labelled as \mintinline[breaklines]{minizinc}{number + 1}) is exactly one. Then with the last one we impose the number of starting cells (i.e. labelled with 12 in the first dimension of the board) to be exactly the same as the number of starting points.
