\section{MiniZinc model}
The approach taken with MiniZinc is completely different. Data structures used to represent the same data and relations are different, input and output format are completely different.
\subsection{Data structures}
Let us introduce everything with order:
\begin{minted}[breaklines, linenos]{minizinc}
    par int: dimension; set of int: D = 1..dimension; 
    par int: number; set of int: N = 1..number; 
    array[N, 1..3] of par 0..dimension: starting_points; 
    array[D, D, 1..2] of var 0..12: Board;
\end{minted}

On line 1 we define the dimension of the board as a parameter named, and a set of integers \mintinline{minizinc}{N} on top of that dimension.
After that, on line 2, we need to define also the number of starting points\footnote{This is something not necessaries in ASP, but it is necessary in MiniZinc as we are using a multidimensional array to represent the starting points and MiniZinc does not support dynamically sized arrays.}. and again a set of int \mintinline{minizinc}{N} on top of that.
On line 3 we define the input format of the starting points which will be of the form:
\begin{minted}[breaklines, linenos, escapeinside=@@]{minizinc}
    starting_points = [|X@$_1$@, Y@$_1$@, T@$_1$@
                         .
                         .
                         .
                       |X@$_n$@, Y@$_n$@, T@$_n$@|];
\end{minted}
with  \mintinline{minizinc}{n} the number of starting points and each line representing, in order, the X, Y coordinates of a starting point and the number of turns it has to make in order to arrive to the goal.
Finally, on line 3, we define the multidimensional array of variables \mintinline{minizinc}{Board}. Board is of size $D$x$D$ and has two dimensions: the first one will store the elements of the board from the one represented in~\ref{figm:board}:
\begin{figure}[ht]
    \centering
    \def\svgwidth{\columnwidth}
    \import{./Images/}{minizinc.pdf_tex}
    \caption{First dimension of Board}
    \label{figm:board}
\end{figure}
the second dimension of the board will keep track of which path\footnote{We have one path for each starting point, all paths intersect in one single point at the end.} each cell belongs to, this is necessary, as we will se later, in order to constraint the number of turns each path as to make.

\subsection{Constraints}


\paragraph{Predicates}

\paragraph{Global Constraints}